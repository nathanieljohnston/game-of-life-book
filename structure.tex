\ifdefined\FORPRINTING
% No watermarking for printed version of book.
\pdfinclusioncopyfonts=1% make sure fonts are embedded even from included figures in the print version of the book
\else
%% WATERMARKING COMMANDS.
\usepackage{draftwatermark}
\SetWatermarkText{Free PDF. \ See conwaylife.com/book for pattern files and print version. \qquad\qquad\qquad\qquad\qquad\qquad\qquad ${}$}
\SetWatermarkColor[gray]{0.75}
\SetWatermarkFontSize{0.5cm}
\SetWatermarkAngle{90}
\SetWatermarkHorCenter{20.1cm}
%% END WATERMARKING
\fi

\ifdefined\FORPRINTING
\usepackage[papersize={222.25mm,285.75mm},top=28mm,inner=33.9mm,bottom=22mm,outer=28mm,headsep=10pt,layout=letterpaper,layouthoffset=3.175mm,layoutvoffset=3.175mm]{geometry}
\else
\usepackage[top=28mm,left=30.95mm,bottom=22mm,right=30.95mm,headsep=10pt,letterpaper]{geometry}
\fi

% BEGIN adjust footnote font family
\usepackage{etoolbox}
\makeatletter
\patchcmd{\@footnotetext}{\footnotesize}{\footnotesize\sffamily}{}{}
\makeatother
% END adjust footnote font family

%\usepackage[top=3cm,bottom=3cm,left=3cm,right=3cm,headsep=10pt,a4paper]{geometry} % Page margins

\newcommand{\bookURL}[1]{http://www.conwaylife.com/book/#1}

\usepackage{graphicx} % Required for including pictures

\graphicspath{{images/}} % Specifies the directory where pictures are stored

\usepackage{tikz} % Required for drawing custom shapes
\usepackage{stackengine}
\usetikzlibrary{arrows}
\usetikzlibrary{shapes.misc}
\usetikzlibrary{decorations.pathreplacing}
\usetikzlibrary{calc,fpu}
\usepackage{setspace}% needed for setstretch in the appendix


\def\mfilbreak{\par\vfil\penalty-20\vfilneg}
\raggedbottom % Needed to avoid ugly vertical stretching on pages. The book template I'm using flushes the page bottoms by default.


% DEFINES A GLIDER SHAPE FOR USE IN TIKZ
\def\glider#1#2#3{
\begin{scope}[shift={#1}, rotate=#2, scale=#3]
	\fill[black](0,0) -- (1,0) -- (1,1) -- (0,1) -- cycle;
	\fill[black](8/7+0,0) -- (8/7+1,0) -- (8/7+1,1) -- (8/7+0,1) -- cycle;
	\fill[black](16/7+0,0) -- (16/7+1,0) -- (16/7+1,1) -- (16/7+0,1) -- cycle;
	\fill[black](16/7+0,8/7+0) -- (16/7+1,8/7+0) -- (16/7+1,8/7+1) -- (16/7+0,8/7+1) -- cycle;
	\fill[black](8/7+0,16/7+0) -- (8/7+1,16/7+0) -- (8/7+1,16/7+1) -- (8/7+0,16/7+1) -- cycle;
\end{scope}}
\def\lwss#1#2#3{
\begin{scope}[shift={#1}, rotate=#2, scale=#3]
	\fill[black](0,0) -- (1,0) -- (1,1) -- (0,1) -- cycle;
	\fill[black](8/7+0,0) -- (8/7+1,0) -- (8/7+1,1) -- (8/7+0,1) -- cycle;
	\fill[black](16/7+0,0) -- (16/7+1,0) -- (16/7+1,1) -- (16/7+0,1) -- cycle;
	\fill[black](24/7+0,0) -- (24/7+1,0) -- (24/7+1,1) -- (24/7+0,1) -- cycle;
	\fill[black](32/7+0,8/7+0) -- (32/7+1,8/7+0) -- (32/7+1,8/7+1) -- (32/7+0,8/7+1) -- cycle;
	\fill[black](0,8/7+0) -- (1,8/7+0) -- (1,8/7+1) -- (0,8/7+1) -- cycle;
	\fill[black](0,16/7+0) -- (1,16/7+0) -- (1,16/7+1) -- (0,16/7+1) -- cycle;
	\fill[black](8/7+0,24/7+0) -- (8/7+1,24/7+0) -- (8/7+1,24/7+1) -- (8/7+0,24/7+1) -- cycle;
	\fill[black](32/7+0,24/7+0) -- (32/7+1,24/7+0) -- (32/7+1,24/7+1) -- (32/7+0,24/7+1) -- cycle;
\end{scope}}
\def\mwss#1#2#3{
\begin{scope}[shift={#1}, rotate=#2, scale=#3]
	\fill[black](0,0) -- (1,0) -- (1,1) -- (0,1) -- cycle;
	\fill[black](8/7+0,0) -- (8/7+1,0) -- (8/7+1,1) -- (8/7+0,1) -- cycle;
	\fill[black](16/7+0,0) -- (16/7+1,0) -- (16/7+1,1) -- (16/7+0,1) -- cycle;
	\fill[black](24/7+0,0) -- (24/7+1,0) -- (24/7+1,1) -- (24/7+0,1) -- cycle;
	\fill[black](32/7+0,0) -- (32/7+1,0) -- (32/7+1,1) -- (32/7+0,1) -- cycle;
	\fill[black](40/7+0,8/7+0) -- (40/7+1,8/7+0) -- (40/7+1,8/7+1) -- (40/7+0,8/7+1) -- cycle;
	\fill[black](0,8/7+0) -- (1,8/7+0) -- (1,8/7+1) -- (0,8/7+1) -- cycle;
	\fill[black](0,16/7+0) -- (1,16/7+0) -- (1,16/7+1) -- (0,16/7+1) -- cycle;
	\fill[black](8/7+0,24/7+0) -- (8/7+1,24/7+0) -- (8/7+1,24/7+1) -- (8/7+0,24/7+1) -- cycle;
	\fill[black](24/7+0,32/7+0) -- (24/7+1,32/7+0) -- (24/7+1,32/7+1) -- (24/7+0,32/7+1) -- cycle;
	\fill[black](40/7+0,24/7+0) -- (40/7+1,24/7+0) -- (40/7+1,24/7+1) -- (40/7+0,24/7+1) -- cycle;
\end{scope}}

\usepackage[english]{babel} % English language/hyphenation
\usepackage{gensymb} % gives the degree symbol

\usepackage{enumitem} % Customize lists
\setlist{nolistsep} % Reduce spacing between bullet points and numbered lists
\usepackage[outline]{contour}

\usepackage{booktabs} % Required for nicer horizontal rules in tables
\usepackage{clock} % Required for generation clock icon
\ClockFrametrue\ClockStyle=3 % Format the clock icons

%% The next commands allow for vertical centering of images next to each other in figures.
\newcommand*{\vcenteredhbox}[1]{\begingroup
\setbox0=\hbox{#1}\parbox{\wd0}{\box0}\endgroup}
\newcommand*{\vcenteredhboxfx}[1]{\begingroup\parbox{0.08\textwidth}{\centering#1}\endgroup}

\usepackage{epigraph}
\setlength{\epigraphwidth}{0.6\textwidth}

\usepackage[font={sf,footnotesize},labelfont=bf]{caption}
\usepackage{subcaption} % allows subfigures within figures
\captionsetup[subfigure]{labelformat=simple,subrefformat=simple,labelsep=space,font={sf,footnotesize}} % Configure subfigure captions in non-float environments
\renewcommand\thesubfigure{(\alph{subfigure})}
\usepackage{wrapfig} % for floating images on the right or left side of the page.
\usepackage{multirow} % Used for some subfigure environments when one image is huge
\usepackage{multicol} % Used for enumerate in exercises with multiple images to put them side by side
\usepackage[bottom]{footmisc} % Forces footnotes to the bottom of the page
%\usepackage{algorithm} % for pseudocode in later chapters of book
\usepackage{algorithmicx} % for pseudocode in later chapters of book
\usepackage{float}
\floatstyle{ruled}
\newfloat{pseudocode}{!htb}{lop}
\floatname{pseudocode}{Pseudocode}
\newfloat{apgsembly}{!htb}{lop}
\floatname{apgsembly}{APGsembly}

\usepackage{xcolor} % Required for specifying colors by name
\usepackage{colortbl}
\interfootnotelinepenalty=10000 % prevent long footnotes from being split across two pages

%----------------------------------------------------------------------------------------
%	FONTS
%----------------------------------------------------------------------------------------

\definecolor{rawocre}{RGB}{0,96,128} % Define the blue color used for highlighting throughout the book
\definecolor{ocre}{RGB}{0,96,128} % Define the blue color used for highlighting throughout the book
\definecolor{linkcol}{RGB}{0,48,64}
\definecolor{gridgray}{RGB}{192,192,192}
\definecolor{medgray}{RGB}{128,128,128}
\definecolor{redback}{RGB}{255,168,168}
\definecolor{redback2}{RGB}{128,0,0}
\definecolor{greenback}{RGB}{168,255,168}
\definecolor{greenpastel}{RGB}{210,255,210}
\definecolor{greenback2}{RGB}{0,192,0}
\definecolor{darkgreenback}{RGB}{0,96,0}
\definecolor{yellowback}{RGB}{255,255,168}
\definecolor{darkyellowback}{RGB}{96,96,0}
\definecolor{orangeback}{RGB}{255,192,128}
\definecolor{orangeback2}{RGB}{255,220,192}
\definecolor{yellowback2}{RGB}{255,255,192}
\definecolor{darkorangeback}{RGB}{128,54,0}
\definecolor{aquaback}{RGB}{192,255,255}
\definecolor{darkaquaback}{RGB}{0,96,96}
\definecolor{blueback}{RGB}{168,168,255}
\definecolor{magentaback}{RGB}{255,210,255}
\definecolor{darkmagentaback}{RGB}{96,0,96}
\newcommand{\cmark}{\checkmark} % checkmark
\newcommand{\xmark}{\ding{55}} % matching X
\newcommand{\thousep}{ \, } % separator (thin space) to use in numbers

\newcommand{\chapterfolder}{cover/}

\usepackage{avant} % Use the Avantgarde font for headings
%\usepackage{times} % Use the Times font for headings
\usepackage{mathptmx} % Use the Adobe Times Roman as the default text font together with math symbols from the Sym­bol, Chancery and Com­puter Modern fonts

\usepackage{microtype} % Slightly tweak font spacing for aesthetics
\usepackage[utf8]{inputenc} % Required for including letters with accents
\usepackage[T1]{fontenc} % Use 8-bit encoding that has 256 glyphs

%----------------------------------------------------------------------------------------
%	BIBLIOGRAPHY AND INDEX
%----------------------------------------------------------------------------------------

\usepackage{calc} % For simpler calculation - used for spacing the index letter headings correctly
\usepackage{imakeidx}
\makeindex[columns=2, title=Index, options= -s StyleInd.ist]

%----------------------------------------------------------------------------------------
%	MAIN TABLE OF CONTENTS
%----------------------------------------------------------------------------------------

\usepackage{titletoc} % Required for manipulating the table of contents

\contentsmargin{0cm} % Removes the default margin

% Part text styling
\titlecontents{part}[0cm]
{\addvspace{20pt}\centering\large\bfseries}
{}
{}
{}

% Chapter text styling
\titlecontents{chapter}[1.25cm] % Indentation
{\addvspace{12pt}\large\sffamily\bfseries} % Spacing and font options for chapters
{\color{ocre!60}\contentslabel[\Large\thecontentslabel]{1.25cm}\color{ocre}} % Chapter number
{\color{ocre}}  
{\color{ocre!60}\normalsize\;\titlerule*[.5pc]{.}\;\thecontentspage} % Page number

% Section text styling
\titlecontents{section}[1.25cm] % Indentation
{\addvspace{3pt}\sffamily\mdseries} % Spacing and font options for sections
{\contentslabel[\thecontentslabel]{1.25cm}} % Section number
{}
{\hfill\color{black}\thecontentspage} % Page number
[]

% Subsection text styling
\titlecontents{subsection}[1.25cm] % Indentation
{\addvspace{1pt}\sffamily\small} % Spacing and font options for subsections
{\contentslabel[\thecontentslabel]{1.25cm}} % Subsection number
{}
{\ \titlerule*[.5pc]{.}\;\thecontentspage} % Page number
[]

% List of figures
\titlecontents{figure}[0em]
{\addvspace{-5pt}\sffamily}
{\thecontentslabel\hspace*{1em}}
{}
{\ \titlerule*[.5pc]{.}\;\thecontentspage}
[]

% List of tables
\titlecontents{table}[0em]
{\addvspace{-5pt}\sffamily}
{\thecontentslabel\hspace*{1em}}
{}
{\ \titlerule*[.5pc]{.}\;\thecontentspage}
[]

%----------------------------------------------------------------------------------------
%	MINI TABLE OF CONTENTS IN PART HEADS
%----------------------------------------------------------------------------------------

% Chapter text styling
\titlecontents{lchapter}[0em] % Indenting
{\addvspace{15pt}\large\sffamily\bfseries} % Spacing and font options for chapters
{\color{ocre}\contentslabel[\Large\thecontentslabel]{1.25cm}\color{ocre}} % Chapter number
{}  
{\color{ocre}\normalsize\sffamily\bfseries\;\titlerule*[.5pc]{.}\;\thecontentspage} % Page number

% Section text styling
\titlecontents{lsection}[0em] % Indenting
{\sffamily\small} % Spacing and font options for sections
{\contentslabel[\thecontentslabel]{1.25cm}} % Section number
{}
{}

% Subsection text styling
\titlecontents{lsubsection}[.5em] % Indentation
{\normalfont\footnotesize\sffamily} % Font settings
{}
{}
{}

%----------------------------------------------------------------------------------------
%	PAGE HEADERS
%----------------------------------------------------------------------------------------

\usepackage{fancyhdr} % Required for header and footer configuration

\pagestyle{fancy}
% \renewcommand{\chaptermark}[1]{\markboth{\sffamily\normalsize\bfseries\chaptername\ \thechapter.\ #1}{}} % Chapter text font settings
\renewcommand{\sectionmark}[1]{\markright{\sffamily\normalsize\thesection\hspace{5pt}#1}{}} % Section text font settings
\fancyhf{} \fancyhead[LE,RO]{\sffamily\normalsize\thepage} % Font setting for the page number in the header
\fancyhead[LO]{\rightmark} % Print the nearest section name on the left side of odd pages
\fancyhead[RE]{\leftmark} % Print the current chapter name on the right side of even pages
\renewcommand{\headrulewidth}{0.5pt} % Width of the rule under the header
\addtolength{\headheight}{2.5pt} % Increase the spacing around the header slightly
\renewcommand{\footrulewidth}{0pt} % Removes the rule in the footer
\fancypagestyle{plain}{\fancyhead{}\renewcommand{\headrulewidth}{0pt}} % Style for when a plain pagestyle is specified

% Removes the header from odd empty pages at the end of chapters
\makeatletter
\renewcommand{\cleardoublepage}{
\clearpage\ifodd\c@page\else
\hbox{}
\vspace*{\fill}
\thispagestyle{empty}
\newpage
\fi}

%----------------------------------------------------------------------------------------
%	THEOREM STYLES
%----------------------------------------------------------------------------------------

\usepackage{amsmath,amsfonts,amssymb,amsthm} % For math equations, theorems, symbols, etc

\newcommand{\intoo}[2]{\mathopen{]}#1\,;#2\mathclose{[}}
\newcommand{\ud}{\mathop{\mathrm{{}d}}\mathopen{}}
\newcommand{\intff}[2]{\mathopen{[}#1\,;#2\mathclose{]}}
\newtheorem{notation}{Notation}[chapter]

% Boxed/framed environments
\newtheoremstyle{ocrenumbox}% % Theorem style name
{0pt}% Space above
{0pt}% Space below
{\normalfont}% % Body font
{}% Indent amount
{\small\bf\sffamily\color{ocre}}% % Theorem head font
{\;}% Punctuation after theorem head
{0.25em}% Space after theorem head
{\small\sffamily\color{ocre}\thmname{#1}\nobreakspace\thmnumber{\@ifnotempty{#1}{}\@upn{#2}}% Theorem text (e.g. Theorem 2.1)
\thmnote{\nobreakspace\the\thm@notefont\sffamily\bfseries\color{black}---\nobreakspace#3\\[0.01cm]}} % Optional theorem note
\renewcommand{\qedsymbol}{$\blacksquare$}% Optional qed square

\newtheoremstyle{blacknumex}% Theorem style name
{5pt}% Space above
{5pt}% Space below
{\normalfont}% Body font
{} % Indent amount
{\small\bf\sffamily}% Theorem head font
{\;}% Punctuation after theorem head
{0.25em}% Space after theorem head
{\small\sffamily{\tiny\ensuremath{\blacksquare}}\nobreakspace\thmname{#1}\nobreakspace\thmnumber{\@ifnotempty{#1}{}\@upn{#2}}% Theorem text (e.g. Theorem 2.1)
\thmnote{\nobreakspace\the\thm@notefont\sffamily\bfseries---\nobreakspace#3\\[0.01cm]}}% Optional theorem note

\newtheoremstyle{blacknumbox} % Theorem style name
{0pt}% Space above
{0pt}% Space below
{\normalfont}% Body font
{}% Indent amount
{\small\bf\sffamily}% Theorem head font
{\;}% Punctuation after theorem head
{0.25em}% Space after theorem head
{\small\sffamily\color{ocre}\thmname{#1}\nobreakspace\thmnumber{\@ifnotempty{#1}{}\@upn{#2}}% Theorem text (e.g. Theorem 2.1)
\thmnote{\nobreakspace\the\thm@notefont\sffamily\bfseries\color{black}---\nobreakspace#3\\[0.01cm]}}% Optional theorem note

% Non-boxed/non-framed environments
\newtheoremstyle{exercisenum}%
{5pt}% Space above
{5pt}% Space below
{\normalfont}% % Body font
{}% Indent amount
{\footnotesize\bf\sffamily\color{ocre}}% % Theorem head font
{\;}% Punctuation after theorem head
{0.25em}% Space after theorem head
{\footnotesize\sffamily\color{ocre}\thmname{#1}\nobreakspace\thmnumber{\@ifnotempty{#1}{}\@upn{#2}}% Theorem text (e.g. Theorem 2.1)
	\thmnote{\nobreakspace\the\thm@notefont\sffamily\bfseries\color{black}---\nobreakspace#3.}} % Optional theorem note

% Non-boxed/non-framed environments
\newtheoremstyle{ocrenum}% % Theorem style name
{5pt}% Space above
{5pt}% Space below
{\normalfont}% % Body font
{}% Indent amount
{\small\bf\sffamily\color{ocre}}% % Theorem head font
{\;}% Punctuation after theorem head
{0.25em}% Space after theorem head
{\small\sffamily\color{ocre}\thmname{#1}\nobreakspace\thmnumber{\@ifnotempty{#1}{}\@upn{#2}}% Theorem text (e.g. Theorem 2.1)
\thmnote{\nobreakspace\the\thm@notefont\sffamily\bfseries\color{black}---\nobreakspace#3.}} % Optional theorem note
\renewcommand{\qedsymbol}{$\blacksquare$}% Optional qed square
\makeatother

% Defines the theorem text style for each type of theorem to one of the three styles above
\newcounter{dummy} 
\numberwithin{dummy}{chapter}
\theoremstyle{ocrenumbox}
\newtheorem{theoremeT}[dummy]{Theorem}
\newtheorem{corollaryT}[dummy]{Corollary}
\newtheorem{propositionT}[dummy]{Proposition}
\newtheorem{exerciseT}{Exercise}[chapter]
\theoremstyle{blacknumex}
\newtheorem{exampleT}{Example}[chapter]
\theoremstyle{blacknumbox}
\newtheorem{vocabulary}{Vocabulary}[chapter]
\newtheorem{definitionT}{Definition}[chapter]
\theoremstyle{ocrenum}
\newcommand{\rawtilde}{\raise.17ex\hbox{$\scriptstyle\mathtt{\sim}$}}

\newcommand{\probdiff}[1]{{\color{gray}[#1/5]}}


\theoremstyle{exercisenum}
\newcounter{dummyb} 
\numberwithin{dummyb}{chapter}
\newtheorem{problem}[dummyb]{\\[-0.7em]\hspace*{-0.3em}}

% STARS beside some problems 
\newtheorem{problemstar}[dummyb]{\\[-0.7em]\raisebox{0.02cm}{\bf\color{ocre}$\ast$}\hspace*{-0.3em}}
\newcommand\itemstar{\stepcounter{enumi}\item[\raisebox{0.01cm}{\bf\color{ocre}$\ast$}\theenumi]}


% This lets us vertically center contents of table cells and use \\ in table cells.
% From http://tex.stackexchange.com/questions/2441/how-to-add-a-forced-line-break-inside-a-table-cell
\newcommand{\specialcell}[2][c]{\begin{tabular}[#1]{@{}c@{}}#2\end{tabular}}
\newcommand{\specialcelll}[2][l]{\begin{tabular}[#1]{@{}l@{}}#2\end{tabular}}

% Evolution arrows for figures.
\usepackage{intcalc}
\newcommand{\genarrow}[1]{\color{black}{$\xrightarrow{\text{\clock{\intcalcDiv{#1}{60}}{\intcalcMod{#1}{60}} #1}}$}}

\newcommand{\fixarrowa}[1]{\color{black}{$\xrightarrow{\text{\makebox[0pt][l]{\clock{\intcalcDiv{#1}{60}}{\intcalcMod{#1}{60}} #1}\phantom{\clock{0}{3} 3}}}$}}
\newcommand{\fixarrowb}[1]{\color{black}{$\xrightarrow{\text{\makebox[0pt][l]{\clock{\intcalcDiv{#1}{60}}{\intcalcMod{#1}{60}} #1}\phantom{\clock{0}{33} 33}}}$}}
\newcommand{\fixarrowc}[1]{\color{black}{$\xrightarrow{\text{\makebox[0pt][l]{\clock{\intcalcDiv{#1}{60}}{\intcalcMod{#1}{60}} #1}\phantom{\clock{0}{33} 333}}}$}}

\newcommand{\phantomarrowa}{$\xrightarrow{\text{\phantom{\clock{0}{3} 3}}}$}
\newcommand{\phantomarrowb}{$\xrightarrow{\text{\phantom{\clock{0}{33} 33}}}$}
\newcommand{\phantomarrowc}{$\xrightarrow{\text{\phantom{\clock{0}{33} 333}}}$}
\newcommand{\phantomarrowz}{$\xrightarrow{\text{\phantom{\clock{0}{3}}}}$}

\newcommand{\gliderarrow}[1]{\color{black}{$\xrightarrow{\text{\includegraphics[width=0.35cm]{glider.png} #1}}$}}

% Commands for making letter grids in proofs
\newcommand{\letternode}[3]{\node[anchor=center] at (#1,#2){\fontsize{18}{18}\sffamily\bfseries\contour{black}{\textcolor{ocre!40}{#3}}};}
\newcommand{\colorletternode}[4]{\node[anchor=center] at (#2,#3){\fontsize{18}{18}\sffamily\bfseries\contour{black}{\textcolor{#1!40}{#4}}};}
\newcommand{\sizecolorletternode}[5]{\node[anchor=center] at (#2,#3){\fontsize{#5}{#5}\sffamily\bfseries\contour{black}{\textcolor{#1!40}{#4}}};} 
\newcommand{\gridbox}[2]{\setlength{\fboxsep}{0mm}
	\setlength{\fboxrule}{#1}\fcolorbox{gridgray}{gridgray}{#2}}

% This fixes some vertical alignment issues in some tables
\usepackage[math]{cellspace}
\cellspacetoplimit 3pt
\cellspacebottomlimit 1pt

% Don't want URLs to display https:// or http://
\newcommand\httpsurl[1]{\href{https://#1}{\nolinkurl{#1}}}
\newcommand\httpurl[1]{\href{http://#1}{\nolinkurl{#1}}}

%----------------------------------------------------------------------------------------
%	DEFINITION OF COLORED BOXES
%----------------------------------------------------------------------------------------

\RequirePackage[framemethod=default]{mdframed} % Required for creating the theorem, definition, exercise and corollary boxes

% Theorem box
\newmdenv[skipabove=7pt,
skipbelow=7pt,
backgroundcolor=black!5,
linecolor=ocre,
innerleftmargin=5pt,
innerrightmargin=5pt,
innertopmargin=5pt,
leftmargin=0cm,
rightmargin=0cm,
innerbottommargin=5pt]{tBox}

% Exercise box	  
\newmdenv[skipabove=7pt,
skipbelow=7pt,
rightline=false,
leftline=true,
topline=false,
bottomline=false,
backgroundcolor=ocre!10,
linecolor=ocre,
innerleftmargin=5pt,
innerrightmargin=5pt,
innertopmargin=5pt,
innerbottommargin=5pt,
leftmargin=0cm,
rightmargin=0cm,
linewidth=4pt]{eBox}	

% Definition box
\newmdenv[skipabove=7pt,
skipbelow=7pt,
rightline=false,
leftline=true,
topline=false,
bottomline=false,
linecolor=ocre,
innerleftmargin=5pt,
innerrightmargin=5pt,
innertopmargin=5pt,
leftmargin=0cm,
rightmargin=0cm,
linewidth=4pt,
backgroundcolor=black!5,
innerbottommargin=5pt]{dBox}	

% Corollary box
\newmdenv[skipabove=7pt,
skipbelow=7pt,
rightline=false,
leftline=true,
topline=false,
bottomline=false,
linecolor=ocre,
backgroundcolor=black!5,
innerleftmargin=5pt,
innerrightmargin=5pt,
innertopmargin=5pt,
leftmargin=0cm,
rightmargin=0cm,
linewidth=4pt,
innerbottommargin=5pt]{cBox}

% Creates an environment for each type of theorem and assigns it a theorem text style from the "Theorem Styles" section above and a colored box from above
\newenvironment{theorem}{\begin{cBox}\begin{minipage}{\linewidth}\begin{theoremeT}}{\end{theoremeT}\end{minipage}\end{cBox}}	
\newenvironment{exercise}{\begin{eBox}\begin{exerciseT}}{\hfill{\color{ocre}\tiny\ensuremath{\blacksquare}}\end{exerciseT}\end{eBox}}				  
\newenvironment{definition}{\begin{dBox}\begin{definitionT}}{\end{definitionT}\end{dBox}}	
\newenvironment{example}{\begin{exampleT}}{\hfill{\tiny\ensuremath{\blacksquare}}\end{exampleT}}		
\newenvironment{corollary}{\begin{cBox}\begin{corollaryT}}{\end{corollaryT}\end{cBox}}			
\newenvironment{proposition}{\begin{cBox}\begin{propositionT}}{\end{propositionT}\end{cBox}}	

%----------------------------------------------------------------------------------------
%	REMARK ENVIRONMENT
%----------------------------------------------------------------------------------------

\newenvironment{remark}{\par\vspace{10pt}\small % Vertical white space above the remark and smaller font size
\begin{list}{}{
\leftmargin=35pt % Indentation on the left
\rightmargin=25pt}\item\ignorespaces % Indentation on the right
\makebox[-2.5pt]{\begin{tikzpicture}[overlay]
\node[draw=ocre!60,line width=1pt,circle,fill=ocre!25,font=\sffamily\bfseries,inner sep=2pt,outer sep=0pt] at (-15pt,0pt){\textcolor{ocre}{R}};\end{tikzpicture}} % Orange R in a circle
\advance\baselineskip -1pt}{\end{list}\vskip5pt} % Tighter line spacing and white space after remark

%----------------------------------------------------------------------------------------
%	FACT ENVIRONMENT
%----------------------------------------------------------------------------------------

\newenvironment{fact}{\par\vspace{10pt} % Vertical white space above the fact
	\begin{list}{}{
			\leftmargin=35pt % Indentation on the left
			\rightmargin=25pt}\item\ignorespaces % Indentation on the right
		\makebox[-2.5pt]{\begin{tikzpicture}[overlay]
			\node[draw=ocre!60,line width=1pt,circle,fill=ocre!25,font=\sffamily\bfseries,inner sep=2pt,outer sep=0pt] at (-15pt,0pt){\textcolor{ocre}{!}};\end{tikzpicture}} % Orange R in a circle
		\advance\baselineskip -1pt}{\end{list}\vskip5pt} % Tighter line spacing and white space after remark

%----------------------------------------------------------------------------------------
%	QUESTION ENVIRONMENT
%----------------------------------------------------------------------------------------

\newenvironment{question}{\par\vspace{10pt} % Vertical white space above the question
	\begin{list}{}{
			\leftmargin=35pt % Indentation on the left
			\rightmargin=25pt}\item\ignorespaces % Indentation on the right
		\makebox[-2.5pt]{\begin{tikzpicture}[overlay]
			\node[draw=ocre!60,line width=1pt,circle,fill=ocre!25,font=\sffamily\bfseries,inner sep=2pt,outer sep=0pt] at (-15pt,0pt){\textcolor{ocre}{?}};\end{tikzpicture}} % Orange R in a circle
		\advance\baselineskip -1pt}{\end{list}\vskip5pt} % Tighter line spacing and white space after question

%----------------------------------------------------------------------------------------
%	SECTION NUMBERING IN THE MARGIN
%----------------------------------------------------------------------------------------

\makeatletter
\renewcommand{\@seccntformat}[1]{\llap{\textcolor{ocre}{\csname the#1\endcsname}\hspace{1em}}}                    
\renewcommand{\section}{\@startsection{section}{1}{\z@}
{-4ex \@plus -1ex \@minus -.4ex}
{1ex \@plus.2ex }
{\normalfont\large\sffamily\bfseries}}
\renewcommand{\subsection}{\@startsection {subsection}{2}{\z@}
{-3ex \@plus -0.1ex \@minus -.4ex}
{0.5ex \@plus.2ex }
{\normalfont\sffamily\bfseries}}
\renewcommand{\subsubsection}{\@startsection {subsubsection}{3}{\z@}
{-2ex \@plus -0.1ex \@minus -.2ex}
{.2ex \@plus.2ex }
{\normalfont\small\sffamily\bfseries}}                        
\renewcommand\paragraph{\@startsection{paragraph}{4}{\z@}
{-2ex \@plus-.2ex \@minus .2ex}
{.1ex}
{\normalfont\small\sffamily\bfseries}}

%----------------------------------------------------------------------------------------
%	PART HEADINGS
%----------------------------------------------------------------------------------------

% numbered part in the table of contents
\newcommand{\@mypartnumtocformat}[2]{%
\setlength\fboxsep{0pt}%
\noindent\colorbox{ocre!20}{\strut\parbox[c][.7cm]{\ecart}{\color{ocre!70}\Large\sffamily\bfseries\centering#1}}\hskip\esp\colorbox{ocre!40}{\strut\parbox[c][.7cm]{\linewidth-\ecart-\esp}{\Large\sffamily\centering#2}}}%
%%%%%%%%%%%%%%%%%%%%%%%%%%%%%%%%%%
% unnumbered part in the table of contents
\newcommand{\@myparttocformat}[1]{%
\setlength\fboxsep{0pt}%
\noindent\colorbox{ocre!40}{\strut\parbox[c][.7cm]{\linewidth}{\Large\sffamily\centering#1}}}%
%%%%%%%%%%%%%%%%%%%%%%%%%%%%%%%%%%
\newlength\esp
\setlength\esp{4pt}
\newlength\ecart
\setlength\ecart{1.2cm-\esp}
\newcommand{\thepartimage}{}%
\newcommand{\partimage}[1]{\renewcommand{\thepartimage}{#1}}%
\def\@part[#1]#2{%
\ifnum \c@secnumdepth >-2\relax%
\refstepcounter{part}%
\addcontentsline{toc}{part}{\texorpdfstring{\protect\@mypartnumtocformat{\thepart}{#1}}{\partname~\thepart\ ---\ #1}}
\else%
\addcontentsline{toc}{part}{\texorpdfstring{\protect\@myparttocformat{#1}}{#1}}%
\fi%
\startcontents%
\markboth{}{}%
{\thispagestyle{empty}%
\begin{tikzpicture}[remember picture,overlay]%
\node at (current page.north west){\begin{tikzpicture}[remember picture,overlay]%	
\fill[ocre!20](0cm,0cm) rectangle (\paperwidth,-\paperheight);
\node[anchor=north] at (4cm,-3.25cm){\color{ocre!40}\fontsize{220}{100}\sffamily\bfseries\@Roman\c@part}; 
\node[anchor=south east] at (\paperwidth-1cm,-\paperheight+1cm){\parbox[t][][t]{8.5cm}{
\printcontents{l}{0}{\setcounter{tocdepth}{1}}%
}};
\node[anchor=north east] at (\paperwidth-1.5cm,-3.25cm){\parbox[t][][t]{15cm}{\strut\raggedleft\color{white}\fontsize{30}{30}\sffamily\bfseries #2}};
\end{tikzpicture}};
\end{tikzpicture}}%
\@endpart}
\def\@spart#1{%
\startcontents%
\phantomsection
{\thispagestyle{empty}%
\begin{tikzpicture}[remember picture,overlay]%
\node at (current page.north west){\begin{tikzpicture}[remember picture,overlay]%	
\fill[ocre!20](0cm,0cm) rectangle (\paperwidth,-\paperheight);
\node[anchor=north] at (4cm,-3.25cm){\color{ocre!40}\fontsize{220}{100}\sffamily\bfseries $\star{}{}$}; 
\node[anchor=south east] at (\paperwidth-1cm,-\paperheight+1cm){\parbox[t][][t]{8.5cm}{
		\printcontents{l}{0}{\setcounter{tocdepth}{1}}}};%
\node[anchor=north east] at (\paperwidth-1.5cm,-3.25cm){\parbox[t][][t]{15cm}{\strut\raggedleft\color{white}\fontsize{30}{30}\sffamily\bfseries #1}};
\end{tikzpicture}};
\end{tikzpicture}}
\addcontentsline{toc}{part}{\texorpdfstring{%
\setlength\fboxsep{0pt}%
\noindent\hfill\protect\colorbox{ocre!40}{\strut\protect\parbox[c][.7cm]{\linewidth-\ecart-\esp}{\Large\sffamily\protect\centering #1\quad\mbox{}}}}{#1}}%
\@endpart}
\def\@endpart{\vfil\newpage
\if@twoside
\if@openright
\null
\thispagestyle{empty}%
\newpage
\fi
\fi
\if@tempswa
\twocolumn
\fi}

%----------------------------------------------------------------------------------------
%	CHAPTER HEADINGS
%----------------------------------------------------------------------------------------

% A switch to conditionally include a picture, implemented by  Christian Hupfer
\newif\ifusechapterimage
\usechapterimagetrue
\newcommand{\thechapterimage}{}%
\newcommand{\chapterimage}[1]{\ifusechapterimage\renewcommand{\thechapterimage}{#1}\fi}%
\def\@makechapterhead#1{%
{\parindent \z@ \raggedright \normalfont
\ifnum \c@secnumdepth >\m@ne
\if@mainmatter
\begin{tikzpicture}[remember picture,overlay]
\node at (current page.north west)
{\begin{tikzpicture}[remember picture,overlay]
\node[anchor=north west,inner sep=0pt] at (0,0) {\ifusechapterimage\includegraphics[width=\paperwidth]{\thechapterimage}\fi};

% White faded box
\draw[anchor=west] (\Gm@lmargin,-9cm) node [line width=2pt,draw=black,fill=white,fill opacity=0.75,inner sep=15pt]{\strut\makebox[22cm]{}};

\newcommand\chpprntadj{0mm}

% Grid at right
% Columns of squares
\foreach \x in {-3,-2,-1,0,1,...,7}
{
	\filldraw[anchor=west,line width=1pt,draw=medgray,fill=white] (\Gm@lmargin+17.825cm+\chpprntadj-\x*0.375cm,-8.25cm-0.375cm) rectangle (\Gm@lmargin+18.2cm+\chpprntadj-\x*0.375cm,-8.25cm);
	\filldraw[anchor=west,line width=1pt,draw=medgray,fill=white] (\Gm@lmargin+17.825cm+\chpprntadj-\x*0.375cm,-8.25cm-0.375cm) rectangle (\Gm@lmargin+18.2cm+\chpprntadj-\x*0.375cm,-8.25cm-2*0.375cm);
	\filldraw[anchor=west,line width=1pt,draw=medgray,fill=white] (\Gm@lmargin+17.825cm+\chpprntadj-\x*0.375cm,-8.25cm-3*0.375cm) rectangle (\Gm@lmargin+18.2cm+\chpprntadj-\x*0.375cm,-8.25cm-2*0.375cm);
	\filldraw[anchor=west,line width=1pt,draw=medgray,fill=white] (\Gm@lmargin+17.825cm+\chpprntadj-\x*0.375cm,-8.25cm-3*0.375cm) rectangle (\Gm@lmargin+18.2cm+\chpprntadj-\x*0.375cm,-8.25cm-4*0.375cm);
}
% Scattered extra cells
\filldraw[anchor=west,line width=1pt,draw=medgray,fill=white] (\Gm@lmargin+17.825cm+\chpprntadj-8*0.375cm,-8.25cm-0.375cm) rectangle (\Gm@lmargin+18.2cm+\chpprntadj-8*0.375cm,-8.25cm);
\filldraw[anchor=west,line width=1pt,draw=medgray,fill=black] (\Gm@lmargin+17.825cm+\chpprntadj-8*0.375cm,-8.25cm-0.375cm) rectangle (\Gm@lmargin+18.2cm+\chpprntadj-8*0.375cm,-8.25cm-2*0.375cm);
\filldraw[anchor=west,line width=1pt,draw=medgray,fill=white] (\Gm@lmargin+17.825cm+\chpprntadj-8*0.375cm,-8.25cm-3*0.375cm) rectangle (\Gm@lmargin+18.2cm+\chpprntadj-8*0.375cm,-8.25cm-4*0.375cm);

\filldraw[anchor=west,line width=1pt,draw=medgray,fill=white] (\Gm@lmargin+17.825cm+\chpprntadj-9*0.375cm,-8.25cm-0.375cm) rectangle (\Gm@lmargin+18.2cm+\chpprntadj-9*0.375cm,-8.25cm-2*0.375cm);
\filldraw[anchor=west,line width=1pt,draw=medgray,fill=black] (\Gm@lmargin+17.825cm+\chpprntadj-9*0.375cm,-8.25cm-3*0.375cm) rectangle (\Gm@lmargin+18.2cm+\chpprntadj-9*0.375cm,-8.25cm-4*0.375cm);

\filldraw[anchor=west,line width=1pt,draw=medgray,fill=white] (\Gm@lmargin+17.825cm+\chpprntadj-10*0.375cm,-8.25cm-0.375cm) rectangle (\Gm@lmargin+18.2cm+\chpprntadj-10*0.375cm,-8.25cm-2*0.375cm);

% Extra black cells
\filldraw[anchor=west,line width=1pt,draw=medgray,fill=black] (\Gm@lmargin+17.825cm+\chpprntadj-6*0.375cm,-8.25cm-0.375cm) rectangle (\Gm@lmargin+18.2cm+\chpprntadj-6*0.375cm,-8.25cm);
\filldraw[anchor=west,line width=1pt,draw=medgray,fill=black] (\Gm@lmargin+17.825cm+\chpprntadj-3*0.375cm,-8.25cm-0.375cm) rectangle (\Gm@lmargin+18.2cm+\chpprntadj-3*0.375cm,-8.25cm);
\filldraw[anchor=west,line width=1pt,draw=medgray,fill=black] (\Gm@lmargin+17.825cm+\chpprntadj-5*0.375cm,-8.25cm-0.375cm) rectangle (\Gm@lmargin+18.2cm+\chpprntadj-5*0.375cm,-8.25cm-2*0.375cm);
\filldraw[anchor=west,line width=1pt,draw=medgray,fill=black] (\Gm@lmargin+17.825cm+\chpprntadj-4*0.375cm,-8.25cm-0.375cm) rectangle (\Gm@lmargin+18.2cm+\chpprntadj-4*0.375cm,-8.25cm-2*0.375cm);
\filldraw[anchor=west,line width=1pt,draw=medgray,fill=black] (\Gm@lmargin+17.825cm+\chpprntadj,-8.25cm-3*0.375cm) rectangle (\Gm@lmargin+18.2cm+\chpprntadj,-8.25cm-2*0.375cm);
\filldraw[anchor=west,line width=1pt,draw=medgray,fill=black] (\Gm@lmargin+17.825cm+\chpprntadj-0.375cm,-8.25cm-3*0.375cm) rectangle (\Gm@lmargin+18.2cm+\chpprntadj-0.375cm,-8.25cm-4*0.375cm);
\filldraw[anchor=west,line width=1pt,draw=medgray,fill=black] (\Gm@lmargin+17.825cm+\chpprntadj-2*0.375cm,-8.25cm-3*0.375cm) rectangle (\Gm@lmargin+18.2cm+\chpprntadj-2*0.375cm,-8.25cm-4*0.375cm);
\filldraw[anchor=west,line width=1pt,draw=medgray,fill=black] (\Gm@lmargin+17.825cm+\chpprntadj-6*0.375cm,-8.25cm-3*0.375cm) rectangle (\Gm@lmargin+18.2cm+\chpprntadj-6*0.375cm,-8.25cm-4*0.375cm);

% Re-draw black outline on box
\draw[anchor=west] (\Gm@lmargin,-9cm) node [line width=2pt,draw=black,inner sep=15pt]{\strut\makebox[22cm]{}};

\draw[anchor=west] (\Gm@lmargin+.4cm,-9.1cm) node {\huge\sffamily\bfseries\color{black}\thechapter. #1\strut};
\draw[anchor=west,line width=2pt,black] (0,-\paperwidth/2) -- (\paperwidth,-\paperwidth/2);
\end{tikzpicture}};
\end{tikzpicture}
\else
\begin{tikzpicture}[remember picture,overlay]
\node at (current page.north west)
{\begin{tikzpicture}[remember picture,overlay]
\node[anchor=north west,inner sep=0pt] at (0,0) {\ifusechapterimage\includegraphics[width=\paperwidth]{\thechapterimage}\fi};

% White faded box
\draw[anchor=west] (\Gm@lmargin,-9cm) node [line width=2pt,draw=black,fill=white,fill opacity=0.75,inner sep=15pt]{\strut\makebox[22cm]{}};

% Grid at right
% Columns of squares
\foreach \x in {-3,-2,-1,0,1,...,7}
{
	\filldraw[anchor=west,line width=1pt,draw=medgray,fill=white] (\Gm@lmargin+17.825cm-\x*0.375cm,-8.25cm-0.375cm) rectangle (\Gm@lmargin+18.2cm-\x*0.375cm,-8.25cm);
	\filldraw[anchor=west,line width=1pt,draw=medgray,fill=white] (\Gm@lmargin+17.825cm-\x*0.375cm,-8.25cm-0.375cm) rectangle (\Gm@lmargin+18.2cm-\x*0.375cm,-8.25cm-2*0.375cm);
	\filldraw[anchor=west,line width=1pt,draw=medgray,fill=white] (\Gm@lmargin+17.825cm-\x*0.375cm,-8.25cm-3*0.375cm) rectangle (\Gm@lmargin+18.2cm-\x*0.375cm,-8.25cm-2*0.375cm);
	\filldraw[anchor=west,line width=1pt,draw=medgray,fill=white] (\Gm@lmargin+17.825cm-\x*0.375cm,-8.25cm-3*0.375cm) rectangle (\Gm@lmargin+18.2cm-\x*0.375cm,-8.25cm-4*0.375cm);
}
% Scattered extra cells
\filldraw[anchor=west,line width=1pt,draw=medgray,fill=white] (\Gm@lmargin+17.825cm-8*0.375cm,-8.25cm-0.375cm) rectangle (\Gm@lmargin+18.2cm-8*0.375cm,-8.25cm);
\filldraw[anchor=west,line width=1pt,draw=medgray,fill=black] (\Gm@lmargin+17.825cm-8*0.375cm,-8.25cm-0.375cm) rectangle (\Gm@lmargin+18.2cm-8*0.375cm,-8.25cm-2*0.375cm);
\filldraw[anchor=west,line width=1pt,draw=medgray,fill=white] (\Gm@lmargin+17.825cm-8*0.375cm,-8.25cm-3*0.375cm) rectangle (\Gm@lmargin+18.2cm-8*0.375cm,-8.25cm-4*0.375cm);

\filldraw[anchor=west,line width=1pt,draw=medgray,fill=white] (\Gm@lmargin+17.825cm-9*0.375cm,-8.25cm-0.375cm) rectangle (\Gm@lmargin+18.2cm-9*0.375cm,-8.25cm-2*0.375cm);
\filldraw[anchor=west,line width=1pt,draw=medgray,fill=black] (\Gm@lmargin+17.825cm-9*0.375cm,-8.25cm-3*0.375cm) rectangle (\Gm@lmargin+18.2cm-9*0.375cm,-8.25cm-4*0.375cm);

\filldraw[anchor=west,line width=1pt,draw=medgray,fill=white] (\Gm@lmargin+17.825cm-10*0.375cm,-8.25cm-0.375cm) rectangle (\Gm@lmargin+18.2cm-10*0.375cm,-8.25cm-2*0.375cm);

% Extra black cells
\filldraw[anchor=west,line width=1pt,draw=medgray,fill=black] (\Gm@lmargin+17.825cm-6*0.375cm,-8.25cm-0.375cm) rectangle (\Gm@lmargin+18.2cm-6*0.375cm,-8.25cm);
\filldraw[anchor=west,line width=1pt,draw=medgray,fill=black] (\Gm@lmargin+17.825cm-3*0.375cm,-8.25cm-0.375cm) rectangle (\Gm@lmargin+18.2cm-3*0.375cm,-8.25cm);
\filldraw[anchor=west,line width=1pt,draw=medgray,fill=black] (\Gm@lmargin+17.825cm-5*0.375cm,-8.25cm-0.375cm) rectangle (\Gm@lmargin+18.2cm-5*0.375cm,-8.25cm-2*0.375cm);
\filldraw[anchor=west,line width=1pt,draw=medgray,fill=black] (\Gm@lmargin+17.825cm-4*0.375cm,-8.25cm-0.375cm) rectangle (\Gm@lmargin+18.2cm-4*0.375cm,-8.25cm-2*0.375cm);
\filldraw[anchor=west,line width=1pt,draw=medgray,fill=black] (\Gm@lmargin+17.825cm,-8.25cm-3*0.375cm) rectangle (\Gm@lmargin+18.2cm,-8.25cm-2*0.375cm);
\filldraw[anchor=west,line width=1pt,draw=medgray,fill=black] (\Gm@lmargin+17.825cm-0.375cm,-8.25cm-3*0.375cm) rectangle (\Gm@lmargin+18.2cm-0.375cm,-8.25cm-4*0.375cm);
\filldraw[anchor=west,line width=1pt,draw=medgray,fill=black] (\Gm@lmargin+17.825cm-2*0.375cm,-8.25cm-3*0.375cm) rectangle (\Gm@lmargin+18.2cm-2*0.375cm,-8.25cm-4*0.375cm);
\filldraw[anchor=west,line width=1pt,draw=medgray,fill=black] (\Gm@lmargin+17.825cm-6*0.375cm,-8.25cm-3*0.375cm) rectangle (\Gm@lmargin+18.2cm-6*0.375cm,-8.25cm-4*0.375cm);

% Re-draw black outline on box
\draw[anchor=west] (\Gm@lmargin,-9cm) node [line width=2pt,draw=black,inner sep=15pt]{\strut\makebox[22cm]{}};

\draw[anchor=west] (\Gm@lmargin+.4cm,-9.1cm) node {\huge\sffamily\bfseries\color{black}#1\strut};
\draw[anchor=west,line width=2pt,black] (0,-\paperwidth/2) -- (\paperwidth,-\paperwidth/2);
\end{tikzpicture}};
\end{tikzpicture}
\fi\fi\par\vspace*{270\p@}}}

%-------------------------------------------

\def\@makeschapterhead#1{%
\begin{tikzpicture}[remember picture,overlay]
\node at (current page.north west)
{\begin{tikzpicture}[remember picture,overlay]
\node[anchor=north west,inner sep=0pt] at (0,0) {\ifusechapterimage\includegraphics[width=\paperwidth]{\thechapterimage}\fi};

% White faded box
\draw[anchor=west] (\Gm@lmargin,-9cm) node [line width=2pt,draw=black,fill=white,fill opacity=0.75,inner sep=15pt]{\strut\makebox[22cm]{}};

% Grid at right
% Columns of squares
\foreach \x in {-3,-2,-1,0,1,...,7}
{
	\filldraw[anchor=west,line width=1pt,draw=medgray,fill=white] (\Gm@lmargin+17.825cm-\x*0.375cm,-8.25cm-0.375cm) rectangle (\Gm@lmargin+18.2cm-\x*0.375cm,-8.25cm);
	\filldraw[anchor=west,line width=1pt,draw=medgray,fill=white] (\Gm@lmargin+17.825cm-\x*0.375cm,-8.25cm-0.375cm) rectangle (\Gm@lmargin+18.2cm-\x*0.375cm,-8.25cm-2*0.375cm);
	\filldraw[anchor=west,line width=1pt,draw=medgray,fill=white] (\Gm@lmargin+17.825cm-\x*0.375cm,-8.25cm-3*0.375cm) rectangle (\Gm@lmargin+18.2cm-\x*0.375cm,-8.25cm-2*0.375cm);
	\filldraw[anchor=west,line width=1pt,draw=medgray,fill=white] (\Gm@lmargin+17.825cm-\x*0.375cm,-8.25cm-3*0.375cm) rectangle (\Gm@lmargin+18.2cm-\x*0.375cm,-8.25cm-4*0.375cm);
}
% Scattered extra cells
\filldraw[anchor=west,line width=1pt,draw=medgray,fill=white] (\Gm@lmargin+17.825cm-8*0.375cm,-8.25cm-0.375cm) rectangle (\Gm@lmargin+18.2cm-8*0.375cm,-8.25cm);
\filldraw[anchor=west,line width=1pt,draw=medgray,fill=black] (\Gm@lmargin+17.825cm-8*0.375cm,-8.25cm-0.375cm) rectangle (\Gm@lmargin+18.2cm-8*0.375cm,-8.25cm-2*0.375cm);
\filldraw[anchor=west,line width=1pt,draw=medgray,fill=white] (\Gm@lmargin+17.825cm-8*0.375cm,-8.25cm-3*0.375cm) rectangle (\Gm@lmargin+18.2cm-8*0.375cm,-8.25cm-4*0.375cm);

\filldraw[anchor=west,line width=1pt,draw=medgray,fill=white] (\Gm@lmargin+17.825cm-9*0.375cm,-8.25cm-0.375cm) rectangle (\Gm@lmargin+18.2cm-9*0.375cm,-8.25cm-2*0.375cm);
\filldraw[anchor=west,line width=1pt,draw=medgray,fill=black] (\Gm@lmargin+17.825cm-9*0.375cm,-8.25cm-3*0.375cm) rectangle (\Gm@lmargin+18.2cm-9*0.375cm,-8.25cm-4*0.375cm);

\filldraw[anchor=west,line width=1pt,draw=medgray,fill=white] (\Gm@lmargin+17.825cm-10*0.375cm,-8.25cm-0.375cm) rectangle (\Gm@lmargin+18.2cm-10*0.375cm,-8.25cm-2*0.375cm);

% Extra black cells
\filldraw[anchor=west,line width=1pt,draw=medgray,fill=black] (\Gm@lmargin+17.825cm-6*0.375cm,-8.25cm-0.375cm) rectangle (\Gm@lmargin+18.2cm-6*0.375cm,-8.25cm);
\filldraw[anchor=west,line width=1pt,draw=medgray,fill=black] (\Gm@lmargin+17.825cm-3*0.375cm,-8.25cm-0.375cm) rectangle (\Gm@lmargin+18.2cm-3*0.375cm,-8.25cm);
\filldraw[anchor=west,line width=1pt,draw=medgray,fill=black] (\Gm@lmargin+17.825cm-5*0.375cm,-8.25cm-0.375cm) rectangle (\Gm@lmargin+18.2cm-5*0.375cm,-8.25cm-2*0.375cm);
\filldraw[anchor=west,line width=1pt,draw=medgray,fill=black] (\Gm@lmargin+17.825cm-4*0.375cm,-8.25cm-0.375cm) rectangle (\Gm@lmargin+18.2cm-4*0.375cm,-8.25cm-2*0.375cm);
\filldraw[anchor=west,line width=1pt,draw=medgray,fill=black] (\Gm@lmargin+17.825cm,-8.25cm-3*0.375cm) rectangle (\Gm@lmargin+18.2cm,-8.25cm-2*0.375cm);
\filldraw[anchor=west,line width=1pt,draw=medgray,fill=black] (\Gm@lmargin+17.825cm-0.375cm,-8.25cm-3*0.375cm) rectangle (\Gm@lmargin+18.2cm-0.375cm,-8.25cm-4*0.375cm);
\filldraw[anchor=west,line width=1pt,draw=medgray,fill=black] (\Gm@lmargin+17.825cm-2*0.375cm,-8.25cm-3*0.375cm) rectangle (\Gm@lmargin+18.2cm-2*0.375cm,-8.25cm-4*0.375cm);
\filldraw[anchor=west,line width=1pt,draw=medgray,fill=black] (\Gm@lmargin+17.825cm-6*0.375cm,-8.25cm-3*0.375cm) rectangle (\Gm@lmargin+18.2cm-6*0.375cm,-8.25cm-4*0.375cm);

% Re-draw black outline on box
\draw[anchor=west] (\Gm@lmargin,-9cm) node [line width=2pt,draw=black,inner sep=15pt]{\strut\makebox[22cm]{}};

\draw[anchor=west] (\Gm@lmargin+.4cm,-9.1cm) node {\huge\sffamily\bfseries\color{black}#1\strut};
\draw[anchor=west,line width=2pt,black] (0,-\paperwidth/2) -- (\paperwidth,-\paperwidth/2);
\end{tikzpicture}};
\end{tikzpicture}
\par\vspace*{270\p@}}
\makeatother

%----------------------------------------------------------------------------------------
%	HYPERLINKS IN THE DOCUMENTS
%----------------------------------------------------------------------------------------

\usepackage{bookmark}
\ifdefined\FORPRINTING
% If printing, links in black
\hypersetup{hidelinks,filebordercolor=red,menubordercolor=red,linkbordercolor=red,backref=true,pagebackref=true,hyperindex=true,colorlinks=true,breaklinks=true,urlcolor=black,bookmarks=true,citecolor=black,linkcolor=black,bookmarksopen=false}
\else
\usepackage{embedfile}
\hypersetup{hidelinks,filebordercolor=red,menubordercolor=red,linkbordercolor=red,backref=true,pagebackref=true,hyperindex=true,colorlinks=true,breaklinks=true,urlcolor=ocre,bookmarks=true,citecolor=redback2,linkcolor=linkcol,bookmarksopen=false}
\fi
\hypersetup{
	pdftitle = {Conway's Game of Life - Mathematics and Construction},
	pdfauthor = {Nathaniel Johnston and Dave Greene}
}
\usepackage{bookmark}
\bookmarksetup{
open,
numbered,
addtohook={%
\ifnum\bookmarkget{level}=0 % chapter
\bookmarksetup{bold}%
\fi
\ifnum\bookmarkget{level}=-1 % part
\bookmarksetup{color=ocre,bold}%
\fi
}
}

% How do I make an attached non-pdf file display like a link?
% (http://tex.stackexchange.com/q/230581)
\makeatletter
\newcommand*{\embeddedfilelink}[2]{%
	\begingroup
	\leavevmode
	\pdfstartlink
	attr{%
		\Hy@setpdfborder
		\ifx\@pdfhighlight\@empty
		\else
		/H\@pdfhighlight
		\fi
		/C[\@filebordercolor]%
	}%
	user{%
		/Subtype/Link%
		/A<<%
		/Type/Action%
		/S/JavaScript%
		/JS(this.exportDataObject({cName: "#1", nLaunch: 2}))%
		>>%
	}%
	\relax
	\Hy@colorlink\@urlcolor#2%
	\close@pdflink
	\endgroup
}
\makeatother


\makeatletter
\@addtoreset{pseudocode}{chapter}% pseudocode counter resets every chapter
\@addtoreset{apgsembly}{chapter}% apgsembly counter resets every chapter
\makeatother
\renewcommand{\thepseudocode}{\thechapter.\arabic{pseudocode}}% pseudocode # is <chapter>.<pseudocode>
\renewcommand{\theapgsembly}{\thechapter.\arabic{apgsembly}}% apgsembly # is <chapter>.<apgsembly>


\newcommand*{\patternlink}[2]{\ifdefined\FORPRINTING #2\else\embeddedfilelink{patterns/\chapterfolder #1.txt}{#2}\fi}
\newcommand*{\embedlink}[2]{\ifdefined\FORPRINTING\else\embedfile{patterns/\chapterfolder #1.txt}\fi\patternlink{#1}{#2}}
\newcommand*{\patternimg}[2]{\includegraphics[scale=#1]{\chapterfolder #2.png}}
\newcommand*{\patternimglink}[2]{\embedlink{#2}{\patternimg{#1}{#2}}}

\newcommand*{\patternimgwidth}[2]{\includegraphics[width=#1]{\chapterfolder #2.png}}
\newcommand*{\patternimglinkwidth}[2]{\embedlink{#2}{\patternimgwidth{#1}{#2}}}


\newcommand*{\bgbox}[2]{\mbox{\protect\vphantom{#2}\hspace{-0.2em}\smash{\colorbox{#1}{\protect\raisebox{0pt}[5pt][0pt]{#2}}}\hspace{-0.25em}}}

\setcounter{collectmore}{-1}% fixes some footnote placements in multi-column environments (in particular in exercise sections)