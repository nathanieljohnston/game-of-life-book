%%%%%%%%%%%%%%%%%%%%%%%%%%%%%%%%%%%%%%%%%%%%%%%%%%%%%%%%%%%%%%%%%%%%%%%%%
%%   CHAPTER: THE 0E0P METACELL
%%%%%%%%%%%%%%%%%%%%%%%%%%%%%%%%%%%%%%%%%%%%%%%%%%%%%%%%%%%%%%%%%%%%%%%%%

\renewcommand{\chapterfolder}{0e0p/}
\chapterimage{cover/0e0p}
\chapter{The 0E0P Metacell}\label{chp:0e0p}\index{0E0P}


\vspace*{-0.4in}
\epigraph{If you couldn't predict what [Life] did then probably that's because it's capable of doing anything.}{John H. Conway}
\vspace*{0.4in}


\noindent The previous chapter introduced universal construction and presented
several adjustable spaceships using these techniques. This chapter explores a
more complex application of universal construction: a self-reproducing cell
which interacts with its neighbours to emulate Conway's Game of Life or a
different cellular automaton on a much larger and slower scale.

As an example application, there is a simple replicator in a similar cellular
automaton, HighLife (B36/S23), discovered in 1994 by Nathan Thompson. This
takes 12 ticks to reproduce into two identical copies of itself:

% TODO include picture

Each of these attempt the same behaviour. The two copies that would otherwise
occupy the same space mutually annihilate, resulting in (at time 24) two copies
separated by a greater distance. By generation 36 there are four copies. More
generally, at generation $12n$, the number of copies is $2^{h(n)}$, where
$h(n)$ is the \emph{Hamming weight} of $n$ -- the number of `1'-bits when $n$
is written in binary.

However, since we're using Thompson's modified rules instead of Conway's
original rules, this is not a valid replicator in Life: when run, it merely
degenerates into a configuration of eight blinkers. To `import' the behaviour
of the HighLife replicator into regular Life, we need a `unit cell' in Life
which emulates HighLife. We could then appropriately arrange twelve copies of
this unit cell in order to create a pattern that faithfully emulates
HighLife's replicator.

This chapter introduces the \emph{0E0P metacell}, a programmable configuration
which can emulate any non-B0 two-state Moore-neighbourhood cellular automaton.
Since it needs to be able to construct multiple copies of itself in different
locations, it uses the same universal construction technology developed in
the previous chapter.

%%%%%%%%%%%%%%%%%%%%%%%%
\section{Rule Emulation}
%%%%%%%%%%%%%%%%%%%%%%%%

Conway's Game of Life is a 2-state 9-neighbour rule: the next state of a cell
depends on the current state of the cell together with the current states of
the eight surrounding cells. Building a metacell that directly emulates Life
would be highly nontrivial for several reasons:

\begin{itemize}
\item \textbf{Quantity of neighbours}: each metacell would need to be able to
construct a copy of itself in up to eight different positions. If both the
north and east neighbours are present, it may be difficult to manoeuvre a
construction arm to build the north-east neighbour.
\item \textbf{Survival}: a cell can live for multiple generations, so its
logic circuitry would need to be reusable. Reusable circuitry is considerably
more expensive than single-use circuitry: compare, for instance, the size of
a boat and a Snark --- the smallest known single-use and reusable stable
reflector, respectively.
\end{itemize}

One of the main ideas behind the 0E0P metacell is that we can emulate any
2-state 9-neighbour rule with an 8-state 4-neighbour rule, where a cell
$(x, y)$ can only be alive at time $t$ if $x + y + t$ is even. This means
that whenever a cell is alive, all four neighbours are dead, so there is
plenty of empty space; moreover, every live cell always dies.

We use the cell states $\{ 0, 1, 2, 3, 4, 5, 6, 7 \}$, where the background
is state 0. In even generations in the emulating rule, the pattern consists
entirely of cells in states $\{ 0, 7 \}$; a live cell at position $(x, y)$
in generation $t$ of the original rule is encoded by a state-$7$ cell at
position $(x + y, y - x)$ in generation $2t$ of the emulating rule.

In odd generations, the pattern consists entirely of cells in states
$\{ 0, 1, 2, 3, 4, 5, 6 \}$. The transition rule for computing generation
$2t + 1$ from generation $2t$ is a fixed function:

$$ h : \{ 0, 7 \}^4 \rightarrow \{ 0, 1, 2, 3, 4, 5, 6 \} $$

specially chosen such that the following function is injective:

$$ H : \{ 0, 7 \}^9 \rightarrow \{ 0, 1, 2, 3, 4, 5, 6 \}^4 $$

% TODO ensure no page break between declaration and definition

$$ \begin{pmatrix}
a_{11} & a_{12} & a_{13} \\
a_{21} & a_{22} & a_{23} \\
a_{31} & a_{32} & a_{33}
\end{pmatrix} \mapsto \begin{pmatrix}
h(a_{11}, a_{12}, a_{21}, a_{22}) & h(a_{12}, a_{13}, a_{22}, a_{23}) \\
h(a_{21}, a_{22}, a_{31}, a_{32}) & h(a_{22}, a_{23}, a_{32}, a_{33})
\end{pmatrix} $$

Moreover, without loss of generality we can let $h(0, 0, 0, 0) = 0$.

The transition rule for computing generation $2t + 2$ from $2t + 1$ is a
function which depends on the rule being emulated:

$$ g : \{ 0, 1, 2, 3, 4, 5, 6 \}^4 \rightarrow \{ 0, 7 \} $$

The injectivity of $H$ means that we can specify the function $g$
such that the composition $g \circ H$ is exactly the transition rule
we want to emulate (up to having renamed the cell states to `0' and `7').
Making the necessary assumption that a dead cell surrounded entirely by
dead cells remains dead, we have $g(0, 0, 0, 0) = 0$. Consequently, we
can define a function:

$$ f : \{ 0, 1, 2, 3, 4, 5, 6, 7 \}^4 \rightarrow
\{0, 1, 2, 3, 4, 5, 6, 7 \} $$

where $f$ coincides with $g$ on the domain of $g$ and coincides with $h$
on the domain of $h$.

% TODO add an exercise to find a suitable function $h$. Up to permuting
% the different cell states, there are 200 different such functions.

The 0E0P metacell can be programmed to emulate any of the $8^{8^4-1}$
zero-preserving 8-state 4-neighbour rules. It takes $2^{35}$ generations
for the 0E0P metacell to run one generation of the 8-state rule (emulated
at a 45-degree angle), and therefore $2^{36}$ generations to emulate one
generation of the 2-state 9-neighbour rule (in the usual orientation).

%%%%%%%%%%%%%%%%%%%%%%%%%%%%%%%%%%%
\section{Structure of the Metacell}
%%%%%%%%%%%%%%%%%%%%%%%%%%%%%%%%%%%

\begin{figure}[htb]
    \centering
    \includegraphics[width=\textwidth]{0e0p/0e0p_schematic.pdf}
    \caption{A schematic for the 0E0P metacell. The symmetrical outer
    \textbf{shell} is highlighted in pastel green, and the \textbf{nucleus}
    is highlighted in yellow. The unhighlighted region between them is the
    \textbf{kernel}.}\label{fig:0e0p_schematic}
\end{figure}

The large-scale structure of the metacell is tripartite, and each of the
three parts is constructed separately. The outer \textbf{shell} has exact
order-4 rotational symmetry, consisting of four spiral arms which propagate
gliders inwards. Only one of these arms is actually used; the other four
exist only to enforce the strict symmetry.

Inside the shell is the \textbf{kernel}, which does not have any symmetry
constraints. The south corner of the kernel contains logic circuitry for
regulating the metacell's lifecycle and computing the state of the metacell
based on the inputs from the four neighbouring cells. The kernel also contains
an output path, shown in crimson in Figure~\ref{fig:0e0p_schematic}, which
can connect to one of the four construction arms (violet, dashed) or to the
input spiral arm of one of the four neighbours. During the operation of the
metacell, the glider stream is redirected between the different possible
outputs by deleting Snarks.

The largest region of the metacell is the \textbf{nucleus}, a huge
boustrophedonic glider loop with a period of exactly $2^{29}$. The nucleus
is populated by over three million gliders, which together encode all of
the construction recipes along with the lookup table for the metacell's rule.

%%%%%%%%%%%%%%%%%%%%%
\section{Memory Tape}
%%%%%%%%%%%%%%%%%%%%%

Stuff.

%%%%%%%%%%%%%%%%%%%%%%%%%
\section{Logic Circuitry}
%%%%%%%%%%%%%%%%%%%%%%%%%

Stuff.

%%%%%%%%%%%%%%%%%%%%%%
\section{Construction}
%%%%%%%%%%%%%%%%%%%%%%

Stuff.

%%%%%%%%%%%%%%%%%%%%%%%%%%%%%%%%
\section{Encoding Other Cellular Automata in Life}
%%%%%%%%%%%%%%%%%%%%%%%%%%%%%%%%

Stuff.

% Talk about other rules here and the neat-o patterns that we can construct out of many copies of 0E0P.


%%%%%%%%%%%%%%%%%%%%%%%%%%%%%%%%
\section{Notes and Historical Remarks}\label{sec:0e0p_history}\index{metacell}
%%%%%%%%%%%%%%%%%%%%%%%%%%%%%%%%

Many metacells---patterns of size larger than $1 \times 1$ that emulate the behavior of a single cell---were constructed prior to 0E0P. The first one was the \emph{p$5760$~metacell}, which was constructed by David Bell in January 1996. This metacell is much smaller and faster than 0E0P, with a period of just $5760$~generations and a bounding box size of just $500 \times 500$ (see Figure~\ref{fig:p5760_metacell}). However, there are numerous trade-offs that make this metacell less useful:\smallskip

\begin{figure}[!htb]
	\centering
	\begin{tikzpicture}
		\node[inner sep=0pt,anchor=south west] at (0,0) {\embedlink{p5760_metacell}{\patternimg{0.15}{p5760_metacell}}};
		
		\draw[white,line width=2.5pt,opacity=0.6](2.63,8.15) circle (0.2);
		\draw[redback2,line width=1pt](2.63,8.15) circle (0.2);
	\end{tikzpicture}
	\caption{The \emph{p$5760$~metacell}. Whether the cell is considered ``alive'' or ``dead'' is determined by the presence or absence of a glider at the location circled in \bgbox{redback}{red}. That glider, if present, is duplicated $8$ times and sent to its neighbors along the $8$ output paths highlighted in \bgbox{aquaback}{aqua}, signaling to them that it is alive. Similarly, neighboring alive cells send their signals to this one along the input paths highlighted in \bgbox{magentaback}{magenta}.}\label{fig:p5760_metacell}
\end{figure}

\begin{itemize}
	\item[1)] It is hard-wired to emulate Life, and cannot easily be modified to emulate most of the $2^{512}$ different non-isotropic Life-like cellular automata.\smallskip
	
	\item[2)] It is not easy to tell at a glance which ``cells'' are alive and which are dead---it is determined by the presence or absence of a single extra glider in the cell---and thus is not interesting to look at from a far-out zoom level.\smallskip
	
	\item[3)] ``Dead'' cells must be placed on the Life plane, which means that, for example, spaceships cannot be emulated by this metacell unless the pattern is infinitely large.\smallskip
\end{itemize}

The first two of these problems were solved by the \emph{OTCA metapixel},\index{OTCA metapixel} which was constructed by Brice Due from late 2005 to mid-2006. While this metacell is a bit larger and slower than the first metacell (it is $2048 \times 2048$ and has period 35,328), it can be used to emulate any of the $2^{18}$ different outer-totalistic Life-like cellular automata. Indeed, built into its circuitry is an easily-adjustable array of eaters that determine how many live neighboring OTCA metapixels should lead to the birth or survival of the current metapixel (see Figure~\ref{fig:otca_metapixel}). 
% Introduce Life-like CA, since no introduction earlier
% mention out of the blue/honey bit/demultiplexer?

% FIGURE HERE, SHOWING PIXEL AND RULE ENCODINGS

The OTCA metapixel also has the remarkable feature that its alive and dead states look, from a distance, like alive and dead cells. This feature is achieved by the ``alive'' version of the cell releasing $43$ pairs of perpendicular lightweight spaceship streams that mutually annihilate each other, thus partially filling in the otherwise empty center of the metacell. For example, arranging a $1 \times 3$ row of ``alive'' metapixels (and a suitably large ``dead'' array of metapixels around its edges) results in a pattern that looks and evolves like a blinker, but 2,048 times as long and wide and 35,328 times as slow (see Figure~\ref{fig:metablinker}).

\begin{figure}[!htb]
	\centering
	\embedlink{metablinker}{\vcenteredhbox{\patternimg{0.104}{metablinker_0}} \vcenteredhbox{\color{black}{$\xrightarrow{\text{\clock{5}{46} 35,328}}$}} \vcenteredhbox{\patternimg{0.104}{metablinker_35328}}}
	\caption{A \emph{meta-blinker}: an arrangement of OTCA metapixels that emulates a blinker, but with period $2 \times 35,328$.}\label{fig:metablinker}\index{meta-blinker}
\end{figure}

The next notable metacell to be constructed was the \emph{p1 megacell},\index{p1 megacell} by Adam P.~Goucher in 2008. This metacell was, again, larger and slower than the metacells that came before it, with a bounding box of $2^{15} \times 2^{15}$ and a period of $2^{24}$. The new features this time that warranted the extra size and delay were twofold:\smallskip

\begin{itemize}
	\item This metacell was built entirely out of stable (p$1$) components like Herschel tracks, with the exception of a single period~$2^{24}$ gun used to regulate its timing.\footnote{This gun could be swapped out for a gun of another period, but periods that are multiples of $2$ help the pattern run quicker under the HashLife algorithm in Life simulation software like Golly.}\index{HashLife}\smallskip
	
	\item All $2^{512}$ non-totalistic Life-like cellular automata can be emulated by this metacell, versus the $2^{18}$ outer-totalistic Life-like cellular automata that can be emulated by the OTCA metapixel.\smallskip
\end{itemize}

% Figure of p1 megacell? Have had a lot of big figures bunched up closely here, so maybe not.

Finally, Adam P.~Goucher spent 2014--2018 constructing the 0E0P metapixel, which solved problem~(3) described earlier---it does not require a background grid of ``dead'' cells to be placed on the Life plane.


%%%%%%%%%%%%%%%%%%%%%%%%%%%%%%%%%
\section*{Exercises \hfill \normalfont\textsf{\small solutions to starred exercises on \hyperlink{solutions_0e0p}{page \pageref{solutions_0e0p}}}}
\label{sec:solutions_0e0p}
\addcontentsline{toc}{section}{Exercises}
\vspace*{-0.4cm}\hrulefill\vspace*{-0.3cm}\footnotesize\begin{multicols}{2}\vspace*{-0.4cm}\raggedcolumns\interlinepenalty=10000
\setlength{\parskip}{0pt}
%%%%%%%%%%%%%%%%%%%%%%%%%%%%%%%%%


\begin{problem}\label{exer:0e0p_ex1}
	An exercise could be placed here.
\end{problem}


\mfilbreak


\begin{problem}\label{exer:0e0p_ex2}
	Another exercise could be placed here.
\end{problem}


%% EXERCISE END COMMANDS
\end{multicols}
\normalsize\vspace*{0.01cm}
%% DONE EXERCISE END COMMANDS
