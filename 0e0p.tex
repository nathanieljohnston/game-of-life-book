%%%%%%%%%%%%%%%%%%%%%%%%%%%%%%%%%%%%%%%%%%%%%%%%%%%%%%%%%%%%%%%%%%%%%%%%%
%%   CHAPTER: SELF-SUPPORTING SPACESHIPS
%%%%%%%%%%%%%%%%%%%%%%%%%%%%%%%%%%%%%%%%%%%%%%%%%%%%%%%%%%%%%%%%%%%%%%%%%

\renewcommand{\chapterfolder}{0e0p/}
\chapterimage{cover/0e0p}
\chapter{The 0E0P Metacell}\label{chp:0e0p}


\vspace*{-0.4in}
\epigraph{If you couldn't predict what [Life] did then probably that's because it's capable of doing anything.}{John H. Conway}
\vspace*{0.4in}


\noindent Nothing yet.


%%%%%%%%%%%%%%%%%%%%%%%%%%%%%%%%
\section{Some First Subsection}
%%%%%%%%%%%%%%%%%%%%%%%%%%%%%%%%

Stuff.

\begin{figure}[!htb]
	\centering
	\includegraphics[width=\textwidth]{0e0p/0e0p_schematic.pdf}
	\caption{A schematic for the 0E0P metacell.}\label{fig:0e0p_schematic}
\end{figure}


%%%%%%%%%%%%%%%%%%%%%%%%%%%%%%%%
\section{Encoding Other Cellular Automata in Life}
%%%%%%%%%%%%%%%%%%%%%%%%%%%%%%%%

Stuff.

% Talk about other rules here and the neat-o patterns that we can construct out of many copies of 0E0P.


%%%%%%%%%%%%%%%%%%%%%%%%%%%%%%%%
\section{Notes and Historical Remarks}\label{sec:0e0p_history}\index{metacell}
%%%%%%%%%%%%%%%%%%%%%%%%%%%%%%%%

Many metacells---patterns of size larger than $1 \times 1$ that emulate the behavior of a single cell---were constructed prior to 0E0P. The first one was the \emph{p$5760$~metacell}, which was constructed by David Bell in January 1996. This metacell is much smaller and faster than 0E0P, with a period of just $5760$~generations and a bounding box size of just $500 \times 500$ (see Figure~\ref{fig:p5760_metacell}). However, there are numerous trade-offs that make this metacell less useful:\smallskip

\begin{figure}[!htb]
	\centering
	\begin{tikzpicture}
		\node[inner sep=0pt,anchor=south west] at (0,0) {\embedlink{p5760_metacell}{\patternimg{0.15}{p5760_metacell}}};
		
		\draw[white,line width=2.5pt,opacity=0.6](2.63,8.15) circle (0.2);
		\draw[redback2,line width=1pt](2.63,8.15) circle (0.2);
	\end{tikzpicture}
	\caption{The \emph{p$5760$~metacell}. Whether the cell is considered ``alive'' or ``dead'' is determined by the presence or absence of a glider at the location circled in \bgbox{redback}{red}. That glider, if present, is duplicated $8$ times and sent to its neighbors along the $8$ output paths highlighted in \bgbox{aquaback}{aqua}, signaling to them that it is alive. Similarly, neighboring alive cells send their signals to this one along the input paths highlighted in \bgbox{magentaback}{magenta}.}\label{fig:p5760_metacell}
\end{figure}

\begin{itemize}
	\item[1)] It is hard-wired to emulate Life, and cannot easily be modified to emulate most of the $2^{512}$ different non-isotropic Life-like cellular automata.\smallskip
	
	\item[2)] It is not easy to tell at a glance which ``cells'' are alive and which are dead---it is determined by the presence or absence of a single extra glider in the cell---and thus is not interesting to look at from a far-out zoom level.\smallskip
	
	\item[3)] ``Dead'' cells must be placed on the Life plane, which means that, for example, spaceships cannot be emulated by this metacell unless the pattern is infinitely large.\smallskip
\end{itemize}

The first two of these problems were solved by the \emph{OTCA metapixel},\index{OTCA metapixel} which was constructed by Brice Due from late 2005 to mid-2006. While this metacell is a bit larger and slower than the first metacell (it is $2048 \times 2048$ and has period 35,328), it can be used to emulate any of the $2^{18}$ different outer-totalistic Life-like cellular automata. Indeed, built into its circuitry is an easily-adjustable array of eaters that determine how many live neighboring OTCA metapixels should lead to the birth or survival of the current metapixel (see Figure~\ref{fig:otca_metapixel}). 
% Introduce Life-like CA, since no introduction earlier
% mention out of the blue/honey bit/demultiplexer?

% FIGURE HERE, SHOWING PIXEL AND RULE ENCODINGS

The OTCA metapixel also has the remarkable feature that its alive and dead states look, from a distance, like alive and dead cells. This feature is achieved by the ``alive'' version of the cell releasing $43$ pairs of perpendicular lightweight spaceship streams that mutually annihilate each other, thus partially filling in the otherwise empty center of the metacell. For example, arranging a $1 \times 3$ row of ``alive'' metapixels (and a suitably large ``dead'' array of metapixels around its edges) results in a pattern that looks and evolves like a blinker, but 2,048 times as long and wide and 35,328 times as slow (see Figure~\ref{fig:metablinker}).

\begin{figure}[!htb]
	\centering
	\embedlink{metablinker}{\vcenteredhbox{\patternimg{0.104}{metablinker_0}} \vcenteredhbox{\color{black}{$\xrightarrow{\text{\clock{5}{46} 35,328}}$}} \vcenteredhbox{\patternimg{0.104}{metablinker_35328}}}
	\caption{A \emph{meta-blinker}: an arrangement of OTCA metapixels that emulates a blinker, but with period $2 \times 35,328$.}\label{fig:metablinker}\index{meta-blinker}
\end{figure}

The next notable metacell to be constructed was the \emph{p1 megacell},\index{p1 megacell} by Adam P.~Goucher in 2008. This metacell was, again, larger and slower than the metacells that came before it, with a bounding box of $2^{15} \times 2^{15}$ and a period of $2^{24}$. The new features this time that warranted the extra size and delay were twofold:\smallskip

\begin{itemize}
	\item This metacell was built entirely out of stable (p$1$) components like Herschel tracks, with the exception of a single period~$2^{24}$ gun used to regulate its timing.\footnote{This gun could be swapped out for a gun of another period, but periods that are multiples of $2$ help the pattern run quicker under the HashLife algorithm in Life simulation software like Golly.}\index{HashLife}\smallskip
	
	\item All $2^{512}$ non-totalistic Life-like cellular automata can be emulated by this metacell, versus the $2^{18}$ outer-totalistic Life-like cellular automata that can be emulated by the OTCA metapixel.\smallskip
\end{itemize}

% Figure of p1 megacell? Have had a lot of big figures bunched up closely here, so maybe not.

Finally, Adam P.~Goucher spent 2014--2018 constructing the 0E0P metapixel, which solved problem~(3) described earlier---it does not require a background grid of ``dead'' cells to be placed on the Life plane.


%%%%%%%%%%%%%%%%%%%%%%%%%%%%%%%%%
\section*{Exercises \hfill \normalfont\textsf{\small solutions to starred exercises on \hyperlink{solutions_0e0p}{page \pageref{solutions_0e0p}}}}
\label{sec:solutions_0e0p}
\addcontentsline{toc}{section}{Exercises}
\vspace*{-0.4cm}\hrulefill\vspace*{-0.3cm}\footnotesize\begin{multicols}{2}\vspace*{-0.4cm}\raggedcolumns\interlinepenalty=10000
\setlength{\parskip}{0pt}
%%%%%%%%%%%%%%%%%%%%%%%%%%%%%%%%%


\begin{problem}\label{exer:0e0p_ex1}
	An exercise could be placed here.
\end{problem}


\mfilbreak


\begin{problem}\label{exer:0e0p_ex2}
	Another exercise could be placed here.
\end{problem}


%% EXERCISE END COMMANDS
\end{multicols}
\normalsize\vspace*{0.01cm}
%% DONE EXERCISE END COMMANDS