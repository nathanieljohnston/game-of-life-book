%%%%%%%%%%%%%%%%%%%%%%%%%%%%%%%%%%%%%%%%%%%%%%%%%%%%%%%%%%%%%%%%%%%%%%%%%
%%   APPENDIX: MISCELLANEOUS MATH
%%%%%%%%%%%%%%%%%%%%%%%%%%%%%%%%%%%%%%%%%%%%%%%%%%%%%%%%%%%%%%%%%%%%%%%%%

\renewcommand{\chapterfolder}{appendices/}
\chapterimage{cover/appendix_math}

\chapter{Mathematical Miscellany}\label{app:math}

In this appendix, we present some of the miscellaneous bits of mathematical knowledge that we need in order to discuss certain concepts or to prove certain theorems regarding the Game of Life in the main text. While none of these topics are particularly deep, they are often taught at scattered places throughout one's math education (or not taught at all), so we summarize them here for ease of reference.


%%%%%%%%%%%%%%%%%%%%%%%%%%%%%%%%%%%%%%%%%%%%%%%%%%%%%%%%%%%%%%%%%%%%%%%%%
%%   SECTION: MODULAR CONGRUENCE
%%%%%%%%%%%%%%%%%%%%%%%%%%%%%%%%%%%%%%%%%%%%%%%%%%%%%%%%%%%%%%%%%%%%%%%%%
\section{Modular Congruence}\label{sec:modular_arithmetic}\index{modular congruence}\index{mod|see {modular congruence}}

When we divide two integers by each other, we get a \emph{quotient} (i.e., the integer result of division) and a \emph{remainder} (i.e., a piece that is left over from the integer division). For example, dividing $7$ by $3$ gives the quotient $q = \lfloor 7/3 \rfloor = 2$ with a remainder of $r = 1$. The number that we divide by ($3$ in this example) is called the \emph{modulus}, and the remainder is always between $0$ (inclusive) and the modulus (exclusive).

Modular congruence is a way of grouping numbers together based on their remainder upon division by a given modulus $n \geq 2$. For example, if the modulus is $n = 3$ then we say that $1$, $4$, $7$, $10$, and so on are all \emph{congruent modulo $3$}, since they all have the same remainder ($1$) upon division by $3$. Equivalently, we say that two integers $x$ and $y$ are \emph{congruent modulo $n$} if $x-y$ is an integer multiple of $n$, and in this case we write
\[
	x \equiv y \pmod{n}.
\]
For example, $7 \equiv 1 \pmod{3}$ since $7 - 1 = 6$ is a multiple of $3$.

Congruence modulo $n$ can be thought of as partitioning the set of integers into $n$ disjoint sets, called \emph{congruence classes}, each consisting of all integers that are congruent to each other. For example, if $n = 2$ then there are two congruence classes: the set of even integers and the set of odd integers. If $n = 3$ then there are three congruence classes:\footnote{Negative integers can be congruent to each other, and do belong to congruence classes. For example, $7 \equiv -5 \pmod{3}$ since $7-(-5) = 12$ is a multiple of $3$, so $7$ and $-5$ belong to the same mod-$3$ congruence class.}
\begin{align*}
	\{\ldots, -6, -3, 0, 3, 6, 9, \ldots\}, \quad \{\ldots, -5, -2, 1, 4, 7, 10, \ldots\}, \quad \text{and} \quad \{\ldots, -4, -1, 2, 5, 8, 11, \ldots\}.
\end{align*}
Since every integer $x$ belongs to (exactly) one of the congruence classes, if we are given another member $y$ of that congruence class then we can find an integer $k$ so that $x = kn + y$.

One of the most prominent uses of modular congruence in the Game of Life arises when making timing and spacing adjustments in Life circuitry. For example, if we knew of some way to delay a signal by $3$~generation, then we could repeat that reaction $n$ times in order to delay the signal by $3n$~generations. If we also knew of methods of delaying that signal by, say, $5$ and $7$ generations, then we could delay it by any (sufficiently large) amount of our choosing: every mod-$3$ congruence class contains either $3$, $5$, or $7$, so every integer can be written in the form $3n$, $3n + 5$, or $3n + 7$.

A slightly more realistic scenario that occurs in the Game of Life makes use of mod-$8$ arithmetic. It is often easy to delay a glider by any multiple of $8$~generations, so to be able to implement arbitrary delays we ``just'' need to find ways of delaying it by amounts that cover the other $7$ congruence classes. We collect glider-delaying reactions like this in two different places in the main text: Tables~\ref{tab:180_degree_one_time_turners} and~\ref{tab:conduit_phase_changers}.


%%%%%%%%%%%%%%%%%%%%%%%%%%%%%%%%%%%%%%%%%%%%%%%%%%%%%%%%%%%%%%%%%%%%%%%%%
%%   SECTION: GREATEST COMMON DIVISOR
%%%%%%%%%%%%%%%%%%%%%%%%%%%%%%%%%%%%%%%%%%%%%%%%%%%%%%%%%%%%%%%%%%%%%%%%%
\section{Greatest Common Divisor and Least Common Multiple}\label{sec:gcd}\index{greatest common divisor}\index{GCD|see {greatest common divisor}}

The \emph{greatest common divisor} (or \emph{GCD} for short) of two non-zero integers $a$ and $b$ is the largest positive integer that both $a$ and $b$ are integer multiples of. We denote the GCD of $a$ and $b$ by $\mathrm{gcd}(a,b)$. For example, $\mathrm{gcd}(6,15) = 3$, since $6$ and $15$ are each multiples of $3$, but they are not each multiples of any larger integer.

Our primary interest in the GCD comes from the fact that it tells us exactly which equations of the form $ax + by = c$ (where $a,b,c$ are given integers, and $x,y$ are variables we are trying to solve for) have integer solutions.\footnote{An equation of this form is called a \emph{linear Diophantine equation}.} In particular, the following theorem answers this question:

% TODO: Add example to illustrate this theorem, as well as GCD/LCM stuff. This is really opaque for this level of book right now.

\begin{theorem}[B\'ezout's identity]\label{thm:linear_diophantine}
	Let $a, b,$ and $c$ be non-zero integers. Then the equation $ax + by = c$ has a solution where both $x$ and $y$ are integers if and only if $c$ is a multiple of $\mathrm{gcd}(a,b)$. Furthermore, if $a$ and $b$ have opposite signs then $x$ and $y$ can be chosen to both be positive.
\end{theorem}

\begin{proof}
	To prove the ``only if'' direction of the theorem, we note that $\mathrm{gcd}(a,b)$ evenly divides each of $a$ and $b$, so for all integers $x,y$ it evenly divides $ax + by$ as well, so $c$ must be a multiple of $\mathrm{gcd}(a,b)$.
	
	To prove the ``if'' direction of the theorem, we suppose that $c$ is a multiple of $\mathrm{gcd}(a,b)$, and we will show that there exist integers $x,y$ such that $ax+by = c$. To this end, let $x^\prime$ and $y^\prime$ be integers that make $ax^\prime + by^\prime$ as small (but positive) as possible. For convenience, define $s := ax^\prime + by^\prime$. When dividing $a$ by $s$, the remainder $r$ is also of the form $r = ax + by$ since it is obtained by subtracting a multiple of $s = ax^\prime + by^\prime$ from $a$. Since the $0 \leq r < s$, and $s$ is the smallest positive number of this form, the only possibility is that $r = 0$. In other words, $s$ evenly divides $a$ (and a similar argument shows that $s$ evenly divides $b$).
	
	It follows that $c/s$ is an integer (since $c$ is a multiple of $\mathrm{gcd}(a,b)$, which is a multiple of $s$), so $x = x^\prime(c/s)$, $y = y^\prime(c/s)$ is a pair of integers satisfying $ax + by = (ax^\prime + by^\prime)(c/s) = s(c/s) = c$, as desired.
	
	To prove the final claim that $x$ and $y$ can be chosen to be positive when $a$ and $b$ have opposite signs, note that if $x$ and $y$ are integers such that $ax + by = c$ then for any integer $k$ it is also the case that $a(x + k(b/\mathrm{gcd}(a,b))) + b(y - k(a/\mathrm{gcd}(a,b))) = c$. By taking $k$ large enough and positive (if $a < 0$ and $b > 0$) or large enough and negative (if $a > 0$ and $b < 0$), this gives us a positive solution to the equation.
\end{proof}

The \emph{least common multiple} (or \emph{LCM} for short) of two positive integers $a$ and $b$ is the smallest integer that is a multiple of both $a$ and $b$. We often denote the LCM of $a$ and $b$ by $\mathrm{lcm}(a,b)$. For example, $\mathrm{lcm}(6,15) = 30$. It is straightforward to verify that $\mathrm{lcm}(a,b) = ab/\mathrm{gcd}(a,b)$ for all $a,b$.

% Explicitly mention that if gcd(a,n) = 1 then the multiples of a (mod n) reach everything. This idea is used in the silverfish and caterpillar rephasers in chapter 10
% Important: introduce term "relatively prime" and put in the index. This term is used in at least one exercise


%%%%%%%%%%%%%%%%%%%%%%%%%%%%%%%%%%%%%%%%%%%%%%%%%%%%%%%%%%%%%%%%%%%%%%%%%
%%   SECTION: BIG-Theta NOTATION
%%%%%%%%%%%%%%%%%%%%%%%%%%%%%%%%%%%%%%%%%%%%%%%%%%%%%%%%%%%%%%%%%%%%%%%%%
\section{Big-$\Theta$ Notation}\label{sec:bigO}

When describing the long-term behaviour of a Life pattern, we often just want to focus on the ``big picture'', while suppressing the ``fine details''. For example, the pattern displayed in Figure~\ref{fig:max} grows extremely quickly, filling the entire Life plane with zebra stripes\index{aebra stripes} as its four corners expand outward. Patterns that fill the plane like this are called \emph{spacefillers},\index{spacefiller} and this one is called \emph{max}.\footnote{Its name is a reference to the fact that its growth rate is maximal---no pattern can expand faster than $c/2$ in each direction (see Theorem~\ref{thm:speed_limits}), and no pattern can tile the plane with a stable configuration that is more dense than zebra stripes (see Theorem~\ref{thm:still_life_density}).}\index{max}

\begin{figure}[!htb]
	\centering
	\embedlink{max}{\vcenteredhbox{\patternimg{0.093}{max_0}} \vcenteredhbox{\genarrow{100}} \vcenteredhbox{\patternimg{0.093}{max_100}}}
	\caption{The \emph{max} spacefiller. Constructed by Tim Coe in October 1995.}\label{fig:max}
\end{figure}

To communicate how quickly this pattern grows, we could of course give an explicit formula for its population $p(t)$ in generation~$t$, which has the following form:
\begin{align}\label{eq:max_population_formula}
	p(t) = \frac{t^2}{4} + \begin{cases}
		21t/2 + 209, & \text{ if } t \equiv 0 \pmod{4} \\
		21t/2 + 215, & \text{ if } t \equiv 2 \pmod{4} \\
		10t + 923/4, & \text{ otherwise}.
	\end{cases}
\end{align}
However, this formula is quite technical and has many details that we typically do not actually care about. Its ``most important'' piece is the $t^2/4$ term at the front---in the long run (i.e., when $t$ is large), that term has the biggest effect on the population. For this reason, we would typically just say that $p(t)$ ``grows like $t^2/4$'', or even that $p(t)$ ``is proportional to $t^2$''. Big-$\Theta$ notation provides a way of making this terminology precise:

\begin{definition}[Big-$\Theta$ Notation]\label{defn:big_theta}
	Suppose $f$ and $g$ are real-valued functions. We say ``$f(x)$ is $\Theta(g(x))$'' if there exist positive scalars $c$, $C$, and $N$ such that
	\[
		cg(x) \leq f(x) \leq Cg(x) \quad \text{whenever} \quad x \geq N.
	\]
\end{definition}

Informally, the phrase ``$f(x)$ is $\Theta(g(x))$'' means that $f$ and $g$ have the same rate of growth as $x$ gets large, ignoring things like scalars and low-order terms. For example, for the population formula $p(t)$ given in Equation~\eqref{eq:max_population_formula}, we can say that $p(t)$ is $\Theta(t^2)$. This is hopefully somewhat intuitive, but it can be made precise by choosing $c = 1/4$, $C = 1/2$, and $N = 57$ in Definition~\ref{defn:big_theta}---see Figure~\ref{fig:max_population_graph}.\footnote{Many other choices of $c$, $C$, and $N$ are possible as well. For example, we could choose any smaller value of $c$ and/or any larger value of $C$ or $N$. We could even choose a smaller value of $C$ as long as we choose a larger value of $N$ to compensate.}

\begin{figure}[!htbp]
	\centering
	\includegraphics[width=\textwidth]{appendices/max_population.pdf}
	\caption{A plot of the population $p(t)$ of the max pattern from Figure~\ref{fig:max} in generation $t$. Since $p(t)$ is between $t^2/4$ and $t^2/2$ when $t \geq 57$, we say that $p(t)$ is $\Theta(t^2)$.}\label{fig:max_population_graph}
\end{figure}

% TODO: Give a couple more basic mathematical examples? Not motivated by Life, but just things like (any quadratic is Theta(x^2)) and stuff like that?